\documentclass[japanese, macos]{KU2}
\usepackage{layouts}
\usepackage{tabularx}
\usepackage[dvipdfmx]{graphicx}
\usepackage{graphicx}
\usepackage{color}
\usepackage{hyperref}

\textheight=22.2cm
%\renewcommand{\tableofcontents}{\pagestyle{empty}\@starttoc{toc}\clearpage\pagestyle{fancy}}
\hoffset=-12mm
\voffset=-6mm

\begin{document}
\fontsize{12}{20} \selectfont

\title{IoT環境における分散台帳技術の性能分析}
\author{渡辺 隆弘}
\supervisor{林 冬惠}
\setdegree{修士}
\maketitle % output title page

\begin{abstract}
概要
\end{abstract}


\tableofcontents

\chapter{はじめに}
近年,「いつでも,どこでも,何でも,誰でも」ネットワークに繋がる「ユビキタスネットワーク社会」が構想されてきた.接続機器として代表的なものとして,従来はパソコンやスマートフォンが挙げられるが,センサーデバイスの普及に伴い,車や家電といった物理機器,建物もネットワークに接続されるようになった.このように様々なデバイスがネットワークに接続されるようになると,それらのデバイス間での情報交換やデータの収集,それに基づく自動化が行われ,新たな付加価値を生むようになる.このような仕組みはInternet of Things(IoT)と呼ばれる仕組みであり,急速に発展している\cite{Atzori2010}\cite{Gubbi2013}.\\
IoTは医療,スマートホーム,運送など幅広い分野での適用が検討されているが,収集,処理するデータの増加に伴い.深刻なセキュリティ,プライバシーの問題を引き起こすことが指摘されている\cite{dorri2017towards}.
この問題を解決するためにBlockchain(BC)を採用することが検討されている.BCはBitcoin\cite{nakamoto2008bitcoin}の中核を為す分散台帳技術のことである.BCはセキュアな取引を保証するが,IoT分野に適用するにあたり以下のような問題点が存在する.

\begin{itemize}
\item 取引の合意形成のアルゴリズムに基づく高いリソース要件
\item ブロックサイズに基づくスケーラビリティの悪さ
\item 合意形成,および二重支払いを防ぐためのアルゴリズムに基づく遅延
\end{itemize}

これらを解決するためにIoT環境に適用する目的の分散台帳技術の研究が行われている.スマートホームを題材とし,BCをベースとした新しいアーキテクチャを構成した研究\cite{dorri2017towards}では,BCを用いた場合と比べ,パケットと処理のオーバーヘッドを大幅に短縮することを実証している.また,他の例には.BCをベースとしない,有向非循環グラフ(DAG)を用いたTangleと呼ばれるグラフ構造に基づくIOTAという新たな分散台帳技術を開発し,取引の速度,規模のシミュレーションを行う研究\cite{popov2016tangle}\cite{kusmierz2017first}が挙げられる.

IoT環境を想定した分散台帳技術について研究が行われる一方で,それらの技術を評価する明確な指標は存在しない.BCの性能を評価する研究\cite{croman2016scaling}などは存在するものの,IoT環境を想定した際にこれらの評価軸を用いることが適切であるとは言えない.
本研究は,既存の種々の分散台帳技術をIoT環境に適用する際,どのような評価指標を用いることが適切であるか議論を行い,その指標に基づき複数の分散台帳技術の性能比較を行う.

本研究の貢献は以下の通りと考える.
\begin{itemize}
\item IoT環境における分散台帳技術の評価指標を提案する.
\item 提案した評価指標に基づき既存の分散台帳を評価し,IoT分野に適用することが適切であるとされる技術の特徴を提示する.
\end{itemize}

本稿の構成は以下の通りである.第2節では,IoTの概略および BCなどの既存の分散台帳技術の紹介を行う.第3節では,評価指標を提示し,その妥当性についての議論を行う.第4節では,定量的な指標を提示する.第5節では,4節で提示した指標に関して分析を行い,その結果を示す.第6節では得られた結果から,IoT環境における分散台帳技術の適用可能性や技術特性などについて議論する.

\chapter{背景}
本章では,IoTについての導入と,本研究で題材とする分散台帳技術についての説明を行う.その後,IoTに分散台帳技術を適用する研究の外観と,その重要性について述べる.

\section{Internet of Things}
Internet of Things(IoT)とは,様々な物理機器などにセンサーやソフトウェアを組み込むことで,情報交換やデータの収集を行えるネットワークを構築する仕組みである.\cite{Vermesan2009}では,アイデンティティ,物理的属性,および仮想パーソナリティ,知的インターフェースを使用し、情報ネットワークにシームレスに統合されている物理的,もしくは仮想的な”モノ”に存在する標準および相互運用可能な通信プロトコルに基づく,自己構成能力を備えた動的なグローバルネットワークインフラストラクチャとして定義されている.\\
\textcolor{red}{IoTのニーズ,目的,実現できるもの,実現されているもの?}\\
IoTは、物理的な世界と仮想的な世界を橋渡しすることで,スマートな都市,スマートな工場,資源管理、交通機関、健康、福利厚生など、多くのアプリケーション分野に影響を与える.しかし、ソフトウェアアプリケーションの中でIoTを活用することは、ネットワーキングからアプリケーション層まで,特に超大規模,極端な異質性,IoTの動的性などの大きな課題を抱えていることが指摘されている\cite{Bouloukakis2016}.
また,世界中において配備されているセンサーの数は急速に増加しており,加速度的に増大していくとされる.相互に接続されたデバイスの数は2013年の時点で90億個であり,2020年には240億個に到達するとみられている\cite{gubbi2013internet}.これらのセンサーは膨大な量のデータを生成しつづけるが,セキュリティ面に対して十分な対策はなされていない.現在のIoTにおいて,セキュリティとプライバシーを提供する方法を提案するにあたり,3つ課題が存在するとDorriら\cite{dorri2017towards}は指摘している.1つ目はリソースの面である.IoTのネットワークにおいてデータを収集するセンサーなどの末端のデバイスは,高度で複雑なセキュリティメソッドを備えていないという問題がある.2つ目は,プライバシーの問題である.IoTにおいて収集されるデータは個人が特定されるようなデータが含まれることが想定される.種々のデータを収集しながらユーザーのプライバシーを保護しなければならない.3つ目は中央化である.クライアントサーバモデルをはじめとする中央集権的なモデルはIoTには不適切であることが多いと指摘されている.中央集権型のモデルを適用した際,単一障害点,多対一トラフィック,スケーラビリティなどの問題を抱えるとされる.\\

\subsection{IoT環境において求められる要件}
IoT環境のシステムに求められる要件は次のようなものが挙げられる.

\begin{itemize}
\item 膨大なデータに対応するスケーラビリティ
\item プライバシー保護の観点でのセキュリティ
\item サイバー攻撃に対し強いこと
\item リアルタイムな処理
\end{itemize}

既存のクライアント・サーバモデルでは,中央のサーバの処理速度が全体のボトルネックになることや,
中央のサーバがトラブル,攻撃などの被害にあった際にシステムが停止してしまうことといった問題点が指摘されており,これらを解決するために,中央集権型のシステムではなく,分散データベースを用いることが提案されてきた.一方で,現在実用されているBCなどのアーキテクチャは処理に時間がかかること,トランザクションの処理速度が遅いこと,トランザクションあたりの手数料が大きいことなどの理由からこれらを直接IoT環境に適用することは困難だとされている.\\
分散台帳技術は,既存のIoT環境のシステムに比べ,セキュリティの面で優れている.一方で,IoT環境の分散台帳技術において求められる要件は,既存の分散台帳技術に求められる要件と重要視される点が必ずしも等しくないと考えられる.前節で提示した評価指標において,本研究はIoT環境における分散台帳技術の評価指標を作成することを考慮すると,性能効率性,スケーラビリティの2点がより重要視されると考えられる.これは,IoT環境においてはBCなどのシステムに比べ即時的な処理が要求されること,ノード数,データ数が非常に膨大になることが想定されるためである.\\

\section{分散台帳}
分散台帳技術(Distributed Ledger Technology)はデータを中央で管理するのではなく,分散させ管理するデータベースであり,複数の端末が同じ情報を保持することで改竄や攻撃への耐性を高め,管理者が存在せずに信頼性を確保するものである.\\
\textcolor{red}{どのような目的,分野で活用されているか,どのような問題を解決するために生まれた技術か?}\\
本研究において分析対象とするいくつかの分散台帳技術を紹介する.

\subsection{Blockchain}
Blockchain(BC)はBitcoin\cite{nakamoto2008bitcoin}の中核を為す分散台帳技術である.ブロックと呼ばれるトランザクション(TX)の集合を一定時間おきに生成し,それを鎖のように連結させていくことでデータを保管するデータベースである.右に行くほど新しいブロックであり,鎖状に連結される.トランザクションと呼ばれるユーザ間での取引は10分おきにブロックという単位にまとめられ,台帳の最新のブロックの直後に連結される.台帳にブロックを追加するためにはProof of Work(PoW)と呼ばれる作業が必要となる.この作業は多くの計算資源を要求するため,データを保証する根拠となる.PoWを行い,ブロックを追加することができたユーザには報酬が与えられ,この一連の作業はマイニングと呼ばれる.BCの各ブロックは,直前のブロックの内容のハッシュ値と,取引データを含んでいる.そのため,台帳の一部を改竄するには,その後のブロックの内容を全て書き換える必要があり,BCの改竄は実質的に不可能とされている.\\
\textcolor{red}{詳細な説明,Bitcoinのものに絞って説明したほうが良さそう}

\subsection{Ethereum}
Ethereum\footnote{https://www.ethereum.org/}は次世代のスマートコントラクトと非中央集権型アプリケーションのプラットフォームと表現されている,独自BC上でアプリケーションを配置し,実行できるプラットフォームである.Ethereumはスマートコントラクトを記述するSolidityという言語に代表されるチューリング完全なプログラミング言語を持ち,ユーザはEthereum上のBCに任意の分散型アプリケーションやスマートコントラクトを記述し実行することが可能である.\\
\textcolor{red}{Blockchainとの差異,スマートコントラクト,gasなどについての詳細な説明,用途}\\

\subsection{IOTA}
IOTA\footnote{https://www.iota.org/}はM2Mマイクロペイメントのために開発された仮想通貨の基盤となる分散台帳技術である.IOTAはBCをIoTに適用する際に課題とされる,スケーラビリティとトランザクションフィーを解決するために開発された.IOTAはBCとは異なり,Tangleと呼ばれる分散台帳アーキテクチャを用いている.Tangleは有向非循環グラフ(DAG)に基づいており,BCのようなブロックの概念は存在しない.\\
IOTAにおいて,ネットワークの参加者はBCと異なり,台帳に任意のタイミングでトランザクションを追加することができる.この際,Tangleに追加されているトランザクション(TX)のうち,2つを選択してPoWを行い,それらのトランザクションの正当性を証明することで,自分のトランザクションをネットワークに追加することが可能になる,図の場合,8のトランザクションをグラフに追加するために5,6の2つのトランザクションを検証したということを表している.このように,自分のトランザクションをグラフに追加することが検証の報酬としてはたらくため,IOTAには取引の手数料が存在しない.また,単位時間あたりのトランザクションが増加しても未承認のトランザクション(グラフの左端のトランザクション)の数は発散せず一定であり,安定性があるとされている\cite{kusmierz2017first}.\\
\textcolor{red}{Tangleの詳細な説明,どのような目的で開発されたか,どのように利用されているのか}\\

\section{IoT環境における分散台帳技術}
IoT環境では,現在より膨大な量のデータのやりとりが行われるようになる.Cromanらは,現在のBCの性能評価を行い,理論的に最大1秒あたり7トランザクションの処理が可能であるものの,実際のBCにおける測定では1秒あたり3.3トランザクション程度の性能であると述べている\cite{croman2016scaling}.一方で,現在のVISAの支払いストリームは毎秒平均で2000トランザクションの処理が行われており,BCのスケーラビリティとの大きな乖離が指摘されている.また,BCは仕様上,10分に1度まとめてトランザクションの合意を行うため,リアルタイムの処理を行うことはできない.IoT,金融などのシステムではリアルタイムの処理が要求されるため,この点においても現状のBCを転用することは困難であると指摘されている.\\
また,\cite{decker2013information}では,台帳を共有するネットワークのメッセージ交換がオーバーヘッドになることを指摘している.各ノードが新規ブロックを受け取るまでの遅延は平均で12.6秒であり,ブロックサイズが大きくなればなるほど,遅延時間も比例して大きくなるとしている.\\
Dorriらは,IoT環境としてスマートホームの環境を想定し,既存のBCをベースにした新たな軽量なアーキテクチャを提案し,セキュリティ,プライバシーに対しBCベースのアーキテクチャの有効性について分析を行った\cite{dorri2017blockchain}.また,パケットオーバーヘッド,時間オーバーヘッド,エネルギー消費の観点から提案アーキテクチャを評価,議論した.\\
\textcolor{red}{詳細化,実際のIoT環境の特徴,DLEの特徴を踏まえて}\\



\chapter{定性的な指標}
本研究は,IoT環境に分散台帳技術を適用する際,それらの性能を比較するための適切な評価指標を提示し,既存のアーキテクチャに対し比較を行い,どのようなアーキテクチャがioT環境に適しているか考察する.既存研究において,BCの性能評価,IoTへの適用可能性や具体的なシナリオにおける評価などは議論されてきたが,IoT環境に分散台帳技術を適用する際どのような性能評価が必要であるかという点の議論は未だ不十分である.分散台帳技術の評価を行うにあたり,ネットワーク,サービスコンピューティングなどの観点も含め,以下表\ref{tb:evaluation}のような評価項目が考えられる.

\begin{table}[t]
    \begin{center}
    \scalebox{0.9}{
    \begin{tabular}{|c|l|c|c|} \hline
        大項目 & 概要 & 小項目 & 関連技術\\ \hline
        性能効率性 & 
         \begin{tabular}{c}
      システムの応答時間および\\処理時間並びにスループッ\\ト速度が要求事項を満足す\\る度合い
      \end{tabular}
       & 処理性能 & 
       \begin{tabular}{c}
       ブロックサイズ\\
       トランザクションサイズ\\
       コンセンサス方式\\
       ブロック生成時間
       \end{tabular} \\ \cline{3-4}
       &  & ネットワーク性能 &
        \begin{tabular}{c}
       ネットワーク環境\\
       ノード分散
       \end{tabular} \\ \cline{3-4}
        &  & ブロック確定性能 &
        \begin{tabular}{c}
       コンセンサス方式\\
       ネットワーク環境
       \end{tabular} \\ \cline{3-4}
        &  & 参照性能 &
        \begin{tabular}{c}
       ノード分散\\
       ネットワーク環境\\
       ブロック構造
       \end{tabular} \\ \hline
       
       スケーラビリティ & 
        \begin{tabular}{l}
      処理速度を向上させられる\\度合い,保持するデータ量\\の増大に対する拡張性の度合\\ い,対応可能ノード数など
      \end{tabular}
       & スループット向上性 & 
       \begin{tabular}{c}
       ブロックサイズ\\
       トランザクションサイズ\\
       コンセンサス方式\\
       ブロック生成時間
       \end{tabular} \\ \cline{3-4}
       &  & ネットワーク性能向上性 &
        \begin{tabular}{c}
       ノード分散\\
       ネットワーク環境\\
       P2Pプロトコル
       \end{tabular} \\ \cline{3-4}
        & 
         & 容量拡張性 &
        \begin{tabular}{c}
       ブロックサイズ\\
       トランザクションサイズ\\
       コンセンサス方式\\
       ブロック生成時間       \end{tabular} \\ \cline{3-4}
        & & ノード数拡張性 &
        \begin{tabular}{c}
       データ容量\\
       コンセンサス方式
       \end{tabular} \\ \hline
       
      信頼性 & 
      \begin{tabular}{l}
      運用操作可能及びアクセス可\\能な度合い
      \end{tabular} & 可用性 &
        \begin{tabular}{c}
       単一障害点の有無\\
       コンセンサス方式
       \end{tabular} \\ \hline
        
        セキュリティ &   
       \begin{tabular}{l}
      アクセスすることを認められ\\たデータのみにアクセスでき\\る度合い
      \end{tabular}  & 機密性 & 
       \begin{tabular}{c}
       アクセス管理\\
       データ秘匿化
       \end{tabular} \\ \cline{2-4}
        &
        \begin{tabular}{l}
      行為が引き起こされたことを\\証明することができる度合い
      \end{tabular} & 否認防止性&
        \begin{tabular}{c}
       コンセンサス方式\\
       \end{tabular} \\ \hline
    \end{tabular}
    }
    \end{center}
    \caption{評価軸 \label{tb:evaluation}}
\end{table}


一般的な性質の分析は主に,それらの構造的特徴を捉えるものである.シミュレーションによる分析に対し,種々のシステムの差異がアーキテクチャ,アルゴリズムなどのどの要素によって生じ得たものか検討する際に要求されるものである.図\ref{tb:evaluation}の関連技術の項目が各分散台帳技術の性能,特性差を生むものであるため,一般的な性質の分析指標としてこれらの項目を取り上げる.

\begin{itemize}
\setlength{\itemsep}{0cm}
\item ネットワーク環境
\item ブロックサイズ
\item トランザクションサイズ
\item コンセンサス方式
\item ブロック生成時間
\end{itemize}

ネットワーク環境は,具体的にはユーザが自由にネットワークに参加することができるかということを指す.IoT環境においては,ノード数が増大していくことが考えられるため,管理者によってネットワークへの参加者が制限されるシステムは適さないと考えられる.しかし,管理者がネットワークに参加するユーザを事前に制限することは,PoWを簡素化できるセキュリティを生むために,処理速度を向上させることができるというメリットもある.ブロックサイズは,台帳のブロック1つあたりの容量を指す.ブロックサイズが大きくなれば,1回のブロック生成によって承認されるトランザクションの数が増え,単位時間あたりのトランザクションの処理速度が増大する.一方で,ブロックサイズが増加すればするほど,ブロックがネットワークを伝播する際の遅延は大きくなる.トランザクションサイズは1取引あたりのデータ量を表す.トランザクションサイズが小さくなれば,1ブロックあたりのトランザクション数が増加し,単位時間あたりのトランザクションの処理速度が増大する.一方トランザクションサイズの減少はハッシュ等の簡略化を引き起こし,セキュリティ面での低下をもたらす.コンセンサス方式は,新たなトランザクションを台帳に追加する際にどのような合意形成の方法を取るかを指す.コンセンサス方式によって,合意形成に必要な時間,必要となるリソース,セキュリティなどが変化すると考えられる.ブロック生成時間は,あるブロックが生成されてから次に新たなブロックが生成されるまでに要する時間である.ブロック生成時間が短くなれば,単位時間あたりにより多くのトランザクションを処理することが可能になる.一方で,ブロック生成時間を短くするということはPoWに要する時間を短くするということと同義であり,検証の時間が短くなればなるほどセキュリティの低下を引き起こす.\\

\chapter{定量的な指標}
本章では,分散台帳技術を評価するための指標のうち,定量的な指標について定義し,詳細に説明する.

\section{速度}
速度(遅延)は〜...\\
以下の2点について評価する.
\subsection{ノード間遅延}
\subsection{承認時間}


\section{コスト}
コストは,1トランザクションを処理するためにどの程度のコストが必要であるかという,Cost Per Confirmed Transactionと定義される.1トランザクションを処理するためのコストは以下の内訳からなる.\\
\textcolor{red}{CPCTの話,論文から ハードとPoWが大きく関係するということ}

\section{スケーラビリティ}
スケーラビリティは処理するデータ量の増大及びネットワークの参加者の増大に対する拡張性の度合いである.生成されるトランザクションのスピードがどの程度であればシステムが正常に動作することができるか,データ量やネットワークの参加者が増大した際にネットワーク遅延がどの程度生じるかという点を指標とすることで,データ及びノードが膨大となるIoT環境を想定した評価指標になると考える.本研究では以下の3点について評価する.
\subsection{キャパシティ}
\subsection{ノード規模}
\subsection{処理達成率}

\chapter{定量的な分析}
本章では,4章で提案した評価指標に基づき,既存の分散台帳技術の性能比較を行う.
評価対象として,2章で紹介した,Blockchain,Ethereum,IOTAの3つのシステムを用いる.
\section{シミュレータの設計}

\section{分析}


\chapter{考察}


\chapter{まとめ}


\createbiblio{ref}

\end{document}